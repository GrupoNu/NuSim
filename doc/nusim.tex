\documentclass[a4paper,fleqn,12pt]{article}
\usepackage[utf8]{inputenc}
\usepackage[english]{babel}

\usepackage[left=2.5cm,right=2.5cm,top=2.5cm,bottom=2.5cm]{geometry}
\usepackage{mathtools}
%\usepackage{amsthm}
\usepackage{amsmath}
%\usepackage{nccmath}
\usepackage{amssymb}
\usepackage{amsfonts}
\usepackage{physics}
%\usepackage{dsfont}
%\usepackage{mathrsfs}

\usepackage{titling}
\usepackage{indentfirst}

\usepackage{bm}
\usepackage[dvipsnames]{xcolor}
\usepackage{cancel}

\usepackage{xurl}
\usepackage[colorlinks=true]{hyperref}

%\usepackage{float}
%\usepackage{graphicx}
%\usepackage{tikz}

%%%%%%%%%%%%%%%%%%%%%%%%%%%%%%%%%%%%%%%%%%%%%%%%%%%%%%%%%%%%%%%%%%%%%%

\newcommand{\s}[1]{\mathcal{#1}}
\newcommand{\cc}[1]{\overline{#1}}
\newcommand{\Eval}[3]{\eval{\left( #1 \right)}_{#2}^{#3}}
\newcommand{\unit}[1]{\; \mathrm{#1}}

\newcommand{\eps}{\epsilon}
\newcommand{\vphi}{\varphi}
\newcommand{\cte}{\text{cte}}

\newcommand{\N}{\mathbb{N}}
\newcommand{\Z}{\mathbb{Z}}
\newcommand{\Q}{\mathbb{Q}}
\newcommand{\R}{\mathbb{R}}
\newcommand{\C}{\mathbb{C}}
\renewcommand{\H}{\s{H}}

\newcommand{\n}{\medskip}
\newcommand{\e}{\quad \mathrm{and} \quad}
\newcommand{\ou}{\quad \mathrm{or} \quad}
\newcommand{\virg}{\, , \;}
\newcommand{\ptodo}{\forall \,}
\renewcommand{\implies}{\; \Rightarrow \;}
%\newcommand{\eqname}[1]{\tag*{#1}} % Tag equation with name

%%%%%%%%%%%%%%%%%%%%%%%%%%%%%%%%%%%%%%%%%%%%%%%%%%%%%%%%%%%%%%%%%%%%%%

\setlength{\droptitle}{-8em}


\title{\huge{$\nu$-sim}}
\author{Mateus Marques}

\begin{document}

\maketitle

\section{Numerical description}

In general:
$$
i \dv{\psi}{t} = \H \psi
\implies
\dv{}{t}
\begin{bmatrix}
\real \\ \imaginary
\end{bmatrix}
=
\begin{bmatrix}
\H_\imaginary & \H_\real \\
-\H_\real & \H_\imaginary
\end{bmatrix}
\begin{bmatrix}
\real \\ \imaginary
\end{bmatrix},
$$
where $\psi = \real + i \imaginary = (\real_1 + i \imaginary_1, \ldots, \real_n + i \imaginary_n)$, $\H = \H_\real + i \H_\imaginary$ being $\real, \imaginary, \H_\real, \H_\imaginary$ real.

The original complex system of $n$ equations becomes a real system with $2n$ equations.

In the case of Neutrino Oscillations in matter, our hamiltonian is always of the form:
$$
\H = \H^0 + \text{diag}\big(V(L), 0, \ldots, 0\big),
$$
where $\H^0$ is constant and $V(L) = \sqrt{2} \, G_F N_e(L)$ is the only parameter that depends on the traveled distance $L$ (time also, because $L = ct = t$). $G_F$ is the Fermi constant and $N_e(L)$ is the solar electron density. The hamiltonian $\H$ is so simple because the only neutrino that interacts with solar matter is the neutrino $\nu_e$ of the electron.

The matrix $\H^0$ is simply given by the sandwich:
$$
\H^0 = U \, M \, U^\dagger,
$$
where $U$ is the neutrino mixing matrix (it's only a unitary matrix, and it has standard \href{https://en.wikipedia.org/wiki/Pontecorvo%E2%80%93Maki%E2%80%93Nakagawa%E2%80%93Sakata_matrix}{parametrization}),
and $M$ is the diagonal matrix corresponding to the mass eigenvalues of the neutrinos in vacuum (it can be simplified, making one of its entries zero).

Now, the algorithm is very simple:
\begin{enumerate}

\item Assume $U, M$ and a table $L \times N_e(L)$ of data as input.

\item Using the following structures of the \href{https://www.gnu.org/software/gsl/doc/html/}{GSL Library}:
\begin{verbatim} gsl_complex, gsl_matrix_complex, gsl_matrix; \end{verbatim}
we easily calculate $\H^0 = \H^0_\real + i\H^0_\imaginary$ by complex matrix operations and then obtain $\H^0_\real$, $\H^0_\imaginary$ by taking the real and imaginary parts with the C macros:
\begin{verbatim}
GSL_REAL, GSL_IMAG.
\end{verbatim}

\item Interpolate $N_e(L)$ using \href{http://www.sns.ias.edu/~jnb/SNdata/sndata.html}{Bahcall's data} and compute $V(L) = \sqrt{2} \, G_F N_e(L)$ for $-R_\odot \leq L \leq R_\odot$, where $R_\odot$ is the solar radius. Here we use the following header and type from GSL:
\begin{verbatim}
#include <gsl/gsl_spline.h>
gsl_interp_steffen,
\end{verbatim}
which by the last \href{https://www.gnu.org/software/gsl/doc/html/interp.html#d-interpolation-example-programs}{1D Interpolation Example} seems to be the best spline for the case.

\item Now we simply solve the following ODE numerically and print the results.
$$
\dv{}{t}
\begin{bmatrix}
\real \\ \imaginary
\end{bmatrix}
=
\begin{bmatrix}
\H_\imaginary & \H_\real \\
-\H_\real & \H_\imaginary
\end{bmatrix}
\begin{bmatrix}
\real \\ \imaginary
\end{bmatrix},
$$
where $\H_\real = \H^0_\real + \text{diag}\big(V(L), 0, \ldots, 0\big)$ and $\H_\imaginary = \H^0_\imaginary$. For this we use the header
\begin{verbatim}
#include <gsl/gsl_odeiv2.h>.
\end{verbatim}

\end{enumerate}

\section{Neutrinos de massa}

Seja $\s{E}$ a base dos neutrinos de interação $e, \mu, \tau$ e $\s{B}(t)$ a base instantânea dos autoestados de massa $\ket{\nu_1(t)}, \ket{\nu_2(t)}, \ket{\nu_3(t)}$. Seja então $W(t) = [ I ]_{\s{E} \to \s{B}(t)}$ a matriz mudança de base de $\s{E}$ para $\s{B}(t)$. Isto significa que $\ket{\nu_i(t)} = [ I ]_{\s{E} \to \s{B}(t)} \ket{\nu_\alpha}$, para $\alpha = e, \mu, \tau$ e $i = 1, 2, 3$.

Seja ainda
$$
\phi(t) =
\begin{bmatrix}
\psi_1(t) \\
\psi_2(t) \\
\psi_3(t)
\end{bmatrix},
$$
onde $\ket{\nu(t)} = \psi_1(t) \ket{\nu_1(t)} + \psi_2(t) \ket{\nu_2(t)} + \psi_3(t) \ket{\nu_3(t)}$ está escrito na base $\s{B}(t)$.

Denotando por $\Lambda(t) = \text{diag}\Big(\lambda_1(t), \lambda_2(t), \lambda_3(t)\Big)$
a matriz diagonal que representa a hamiltoniana na base $\s{B}(t)$, temos então que:
$$
i \dv{\phi(t)}{t} = \qty[ - i W(t) \dv{W^\dagger(t)}{t} + \Lambda(t) ] \phi(t).
$$
Esta é a mesma equação (9) que o Boechat obteve, antes de substituir os termos.

\n

O problema com ela são os termos $W(t), \dv{W^\dagger(t)}{t}$, que dificultam numericamente. A dificuldade que eu acho que torna essa EDO intratável é que a diagonalização do GSL \textbf{não é necessariamente contínua}. Com isso, quero dizer que os autovetores podem diferir por uma constante multiplicativa complexa. Veja os \href{https://www.gnu.org/software/gsl/doc/html/eigen.html#examples}{exemplos do GSL}.

\n

Caminhos:
\begin{itemize}
\item Buscar algo mais analítico? A \href{https://en.wikipedia.org/wiki/Pontecorvo%E2%80%93Maki%E2%80%93Nakagawa%E2%80%93Sakata_matrix}{mixing matrix} é muito feia.
\item O que eu tentei fazer? Resolver a EDO na base de interação e mudar para a base $\s{B}(t)$ a todo instante.
\end{itemize}

\end{document}
